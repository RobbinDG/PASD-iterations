\documentclass{article}
% !TeX spellcheck = en_US 
\usepackage[margin=2cm]{geometry}
\usepackage{makecell}
\usepackage{graphicx}
\usepackage{float}
\usepackage{caption}
\usepackage[shortlabels]{enumitem}
\usepackage[bottom]{footmisc}
\usepackage{verbatim}
\usepackage[english]{babel}

\begin{comment}
TODOs
All:
 - Think about how the printouts can be digitalized without having e.g. the police access the system.
 - Name design considerations for the DM
 - Update user wishes, stories and the like to fit the new requirements
 - Power tree and Domain Model need updating according to the new requirements
 - In the class diagram, make sure all visibility is logical (currently, everything is public, marked with +.) See slides for definitions of +, -, #, ~ etc...

Robbin:
 x Update SSD
 x One UC should specify one session. YOu might add more than 1 item per session
 x Check reply from customer
 x A2 Special Overflows.

Oli:
 - Update SSD
 x Specify in UC title what you are updating
 x One UC should specify one session. YOu might update more than 1 item per session
 x Stakeholders and interests box overflows on the left. Use \makecell[l]{text1//text2} for multiline text in tabular
 x ``Frequency of occurence: Frequent'' <-- seriously?!
 x Specify what transaction you are starting and with what parameters
 x ``information'' is ambiguous and non-specific. Use parameters instead.
 x Updated bb-SD B1
 - Usecase B2
 - bb-SD B2
 - gb and wb for C1
 - gb and wb for A2
  
Nicu:
 - Update SSD
 - Specify in UC title what you are deleting
 - One UC should specify one session. YOu might delete more than 1 item per session
 - Honestly, just check the pictures. There's so many comments and remarks.
\end{comment}

\DeclareUnicodeCharacter{20AC}{-}
\begin{document}

\setcounter{secnumdepth}{0}

\hbox{
	\hspace*{0.08\textwidth}
	\rule{0.5pt}{\textheight}
	\hspace*{0.001\textwidth}
	\rule{0.5pt}{\textheight}
	\hspace*{0.05\textwidth}
	\parbox[b]{0.75\textwidth}{
		{\noindent\huge\bfseries Problem Analysis \& Software Design \\}\\[2\baselineskip] % Title
		\large{
		\textbf{Group:} 9\\[\baselineskip]
		\begin{tabular}{@{}l l l l@{}}
			\textbf{Names:}& (A) Robbin de Groot  &\textbf{S-number:} &s3376508\\
			& (B) Oliver Strik  & &s3100693\\
			& (C) Nicu Ghidirimschi & & s3197395
		\end{tabular}}\\

		\vspace{6pt}
		{\large \textbf{Report:} Iteration 2 } \\[\baselineskip] % Tagline or further description
		{\large \textbf{Date:} \today } \\[4\baselineskip] % Tagline or further description
		{\large \textsc{ Prof. E.O. de Brock \& Dr. R. Smedinga }}
	\vspace{0.40\textheight} % Whitespace between the title block and the publisher
	{\Large\noindent \\ Computing Science - Year 2 \vspace{0.7cm}
	\Large\noindent \\University of Groningen \vspace{5pt}\\ \large Faculty of Science \& Engineering }}
}
\newpage

%%% index page
\tableofcontents

\section{Introduction}
An auctioning company called ``The AuctionHouse\textsuperscript{TM}'' auctions provided goods to buyers. Currently, they auction and display the goods in a warehouse just outside of city limits. Owner John wants to automate the administration of auctions and other activities using an IT solution.

\section{Expectations Summary and Conclusion}
John wants the sellers to be able to register their goods in the to-develop-system. These goods then need to be assessed and possibly removed if they lack the requirements. A couple of days before an auction, potential buyers must be able to view the goods. The goods are then auctioned at location (so not through the system).\\
Currently, regular customers get mail informing them of the goods on sale, rather than having to go and see the available goods in person.\\
Payments are done through cash or card, and not through credit cards. Bigger customers get offered a special billing procedure.
The police is handed a list of goods on auctions, so they can identify any stolen goods.
Once the system is completed, a system administrator should make sure every person has the right permissions for the system, and verify that it is operating properly.

\section{Potential users and user wishes}
\subsection{Actors and Users}
What follows is a list of user (groups) that need to interact with the system directly. The selection is derived from the above provided summary. Also is a brief description of the wishes in customer language.
\begin{itemize}[noitemsep]
	\item Owner of The AuctionHouse\textsuperscript{TM} (John)\\
		The owner needs to be able to manage the staff members and their access and permissions in the system.
	\item Private Individuals and Merchants (Owners of the goods)\\
		Sellers need to be able to present their goods to the purchasing agent so he/she can decide whether they can be auctioned.
	\item Purchasing agent\\
		The purchasing agent needs to evaluate the presented goods for possible sale, and if they will be auctioned, needs to make sure they are being stored in the storehouse (which does not necessarily have to be done through the system).
	\item Viewers\\
		Viewers want to see the products that are going to be auctioned, and therefore need to have access to the list of goods.
	\item System Administrator\\
		The system administrator is responsible for the system's behaviour and needs to monitor system activity.
\end{itemize}

\subsection{Other Stakeholders}
Below is the list of stakeholders; people who have interests in the development of the system, or are otherwise involved with it, while not having to interface with it directly or having additional needs compared to other user groups.
\begin{itemize}[noitemsep]
	\item Regular Customer\\
		Regular customers have no extra needs or wishes. However, due to their regular purchases, they receive a list of available items per mail.
	\item Big Customer\\
		Big customers are essentially normal customers, but since their expenses are greater, they are offered a special billing procedure.
	\item Police\\
		The police has no need to directly access the system. However, they are offered a list (printout) of goods to follow potentially stolen goods.
\end{itemize}

\subsection{User Wishes and Stories}
Users and stakeholders need the system to be able to handle their requests. Below is a list of those wishes.
\begin{itemize}[noitemsep]
	\item Administrators: do everything below under test environments
	\item Owner: add/remove/modify/view staff members
	\item Purchasing Agent: register/modify items for sale
	\item Auctioneer: mark item as sold/junk and delete item from the item list
	\item Secretary: Registers buyers and sellers to the system
	\item Secretary: Generate printouts of items for sale/sold/etc.
	\item Secretary: Modify basic parameters of items (e.g. modify the auction date of an item).
	\item Seller: view items they have for sale
	\item Seller: view items they have sold
	\item Buyer: view items they have bought
	\item Public: view items that are for sale
	\item Public: view auction schedule
\end{itemize}
To visualize this, a diagram is provided. This diagram is a ``power tree'' that shows how permissions are divided and who can do the same as another user group.
\begin{figure}[H]
	\centering
	% ignore the compilation warning, it has to do with the pdf and the graphicx package not accounting for some variable within the file, even though the variable has had no purpose for about 10 years now. #research
	\includegraphics[scale=.75]{power_tree.pdf}
	\caption*{A power tree showing the relations between user groups and permissions. Arrows indicate what permissions are inherited. For example, the Owner, Purchasing Agent, Auctioneer, etc. can all do what the public can, with some class specific permissions.}
\end{figure}




\section{Use Cases}
% layout and sectioning from Slides week 1, page 25
\subsection{Create and add an item - (A1)}
\subsubsection*{Analysis}
\begin{tabular}{@{}l l}
\textbf{Scope}:&The AuctionHouse\textsuperscript{TM} automated administration system\\
\textbf{Level}:&User goal\\
\textbf{Primary Actor}:&Purchasing Agent\\
\textbf{Stakeholders and Interests}:&\begin{tabular}[t]{@{}l}
	Purchasing Agent: wants to register an item easily and quickly.\\
	Auctioneer: wants the item to be registered for an upcoming auction.\\
	Seller: wants his goods to be visible to potential buyers.\\
	Buyer: wants to be notified if they are currently have a search request on this item.\\
	Administration System: wants the addition of the item to be recorded.\\
	Search Request System: wants the catalogue the item belongs to be\\known for lookup\end{tabular}\\
\textbf{Preconditions}:&Purchasing agent is identified and authenticated.\\
\textbf{Postconditions}:&\begin{tabular}[t]{@{}l}
	The item has been registered to the system, or a failure was returned.\\
	The product has been registered for an upcoming auction.\\
	The item is visible for potential buyers and the otherwise authorized.\\
	Buyers with a search request on this item are notified of the addition.\\
	The addition of the item was recorded in the administration system.\\
	The catalogue of the item is know to the system.
	\end{tabular}\\
\textbf{Special requirements}:&A list of categories and a list of future auction dates needs to available in the system\\
\textbf{Frequency of occurence}:& \begin{tabular}[t]{@{}l}Depedent on the amount of sellers and auctions per month.\\Since there is 1 auction per month. If we expect about 30 sellers, then there\\are 30 occurences per month, so about 1 per day.\end{tabular}
\end{tabular}\\\\
\textsl{Main Success Scenario}
\begin{enumerate}[noitemsep]
	\item The user starts the `add item' transaction with the system, having all parameters of the item to be added ready.
	\item The system provides the user with a list of categories to choose from, and a list of future planned aution dates.
	\item The user provides all necessary parameters and chooses a category for the item, and an auction date. The parameters to provide are:
	\begin{itemize}[noitemsep]
		\item The amount of the item available
		\item The type of item
		\item A description
		\item If the item possesses any antiquarian value
		\item A minimum price decided by the owner
		\item The date when brought in
		\item Name and address of the owner plus identification
		\item Planned auction date
		\item Distinguishing features
	\end{itemize}
	\item The system creates the item with all provided parameters to the list and generates an Item ID unique to the item.
	\item The system checks if there are any search requests looking for this specific item, and notifies the buyers that issued the request that the item is now available.
	\item The system records the addition of the item in the administration system.
	\item The system saves the catalogue of the item derived from the description.
	\item The system returns to the user with a confirmation message containing the generated item ID.
\end{enumerate}
\textsl{Extensions}
\begin{itemize}[noitemsep]
	\item 3a. If the provided parameters are provided, the system does not continue to add the item, but asks the user to provide the remaining information.
	\item 3b. If an identical item is already registered in the system, the system terminates the transaction, leaving the system state unchanged.
	\item 4a. If the system fails to add the item to the list (perhaps due to internal/system limitations), the process is aborted and the transaction is terminated, rolling back the system state to the most recent stable (when no such problems occured) version.
	\item 7a. If the catalogue the item belongs to is already known in the system, it will not be added.
\end{itemize}
\textsl{System Sequence Diagram}
\begin{figure}[H]
	\centering
	\includegraphics[scale=1]{uml/SD-bb-create.pdf}
	\caption*{Interactions displayed in a System Sequence Diagram defined by the MSS and its extensions in blackbox format}
\end{figure}
\newpage
\textsl{Glossary}
\begin{center}\begin{tabular}{l|l}
\textbf{Function}&\textbf{Clarification}\\\hline\hline
startAddItemTransaction()&\begin{tabular}[t]{@{}l}Starts the `add item' transaction with the system.\end{tabular}\\\hline
addItem(item parameters)&\begin{tabular}[t]{@{}l}Requests the system to add an item, to be created\\from the provided parameters. This request will \underline{not}\\register the item, only ask the system to create and register\\the item. Therefore, if not all parameters are provided,\\this request can return a failure, leaving the system\\state unchanged.\end{tabular}\\\hline
createItem(item parameters)&\begin{tabular}[t]{@{}l}Creates an item instance from provided parameters and\\registers it to the system. Returns the id of the new\\item and the status of the creation. This can be ``success''\\or any kind of internal failure.\end{tabular}\\\hline
saveItemCatalogue(description)&\begin{tabular}[t]{@{}l}Creates an item catalogue from the provided and saves it\\to the list of item catalogues.\end{tabular}\\\hline
notifyInterestedCustomers(ItemID)&\begin{tabular}[t]{@{}l}Sends an email to all buyers who currently have a search\\request looking for this specific item containing a message\\that tells them the item is now available.\end{tabular}\\\hline
recordAddition(ItemID, item parameters)&\begin{tabular}[t]{@{}l}Records the addition of the item to the system in\\the administration system.\end{tabular}
\end{tabular}\end{center}
\subsection{Create and add a user - (A1)}
\subsubsection*{Analysis}
\begin{tabular}{@{}l l}
\textbf{Scope}:&The AuctionHouse\textsuperscript{TM} automated administration system\\
\textbf{Level}:&User goal\\
\textbf{Primary Actor}:&Owner \& Secretary\\
\textbf{Stakeholders and Interests}:&\begin{tabular}[t]{@{}l}Owner: wants to register his staff to the system.\\Secretary: wants to register buyers and sellers to the system.\\Staff, Buyers and Sellers: Want to be granted access to the system.\end{tabular}\\
\textbf{Preconditions}:&User is identified and authenticated.\\
\textbf{Postconditions}:&\begin{tabular}[t]{@{}l}A new user has been registered to the system.\\The person for whom a new user has been added received\\the necessary authorization credentials for the system.\end{tabular}\\
\textbf{Special requirements}:&A list of categories and a list of future auction dates needs to available in the system\\
\textbf{Frequency of occurence}:& \begin{tabular}[t]{@{}l}Depedent on the amount of job applications and new buyers and sellers per month.\\Since not many staff members at required, the former will be rare, say 1 per month.\\The latter will be more common, say 10 times per month. That is an estimated\\11 times per month.\end{tabular}
\end{tabular}\\\\
\textsl{Main Success Scenario}
\begin{enumerate}[noitemsep]
	\item The user starts the `add user' transaction with the system, having all parameters of the user to be added ready.
	\item The system confirms the interaction and asks the user to provide all the necessary parameters.
	\item The user provides all necessary parameters. The parameters to provide are (optional parameters are marked with a `*'):
	\begin{itemize}[noitemsep]
		\item The name of the new user (username)
		\item The class of the new user (e.g. Staff member, Buyer, Seller)\\
		\quad The selection depends on the user doing the transaction
		\item The address of the person bound to the new user
		\item * An email address (for digital mail)
		\item * A bank account number for deposits/sallary
	\end{itemize}
	\item The system creates the user with all provided parameters and adds it to the list. It also generates authentication credentials for the person bound to the new user.
	\item The system returns to the user with a confirmation message containing the generated authentication credentials, or a failure.
\end{enumerate}
\textsl{Extensions}
\begin{itemize}[noitemsep]
	\item When a failure is returned, the system state remains unchanged; no user or other parameters was added.
	\item When not all required parameters were provided, the system will first ask the user to provide it until all parameters is provided.
	\item The system will only provide the classes for the new user based on the class of the user who is doing the transaction.
\end{itemize}
\textsl{System Sequence Diagram}
\begin{figure}[H]
	\centering
	\includegraphics[scale=1]{SD-bb-createuser.pdf}
	\caption*{Interactions displayed in a System Sequence Diagram defined by the MSS and its extensions in blackbox format}
\end{figure}

\subsection{Updating an item - (B1)}
\subsubsection*{Analysis}
\begin{tabular}{@{}l l}
\textbf{Scope}:&The AuctionHouse\textsuperscript{TM} automated administration system\\
\textbf{Level}:&User goal\\
\textbf{Primary Actor}:&Purchasing Agent, Auctioneer, Secretary\\
\textbf{Stakeholders and Interests}:&\begin{tabular}[t]{@{}l}Purchasing Agent, Auctioneer, Secretary: wants to update item parameters quickly and easly.\end{tabular}\\
\textbf{Preconditions}:&Actor is identified and authenticated.\\
\textbf{Postconditions}:&\begin{tabular}[t]{@{}l}The item has been updated on the system, or a failure was returned.\\The updates to the item are (or will become) visible for potential buyers\\and the otherwise authorized.\end{tabular}\\
\textbf{Special requirements}:&\begin{tabular}[t]{@{}l}The current information of an item and a list of future auction dates needs to available\\in the system.\end{tabular}\\
\textbf{Frequency of occurence}:&Frequent\\
\end{tabular}\\\\
\textsl{Main Success Scenario}
\begin{enumerate}[noitemsep]
	\item The user starts the transaction.
	\item The system provides the user with the current information of the item that they are authorised to edit.
	\item The user provides the updated information. The information that can be updated is
	\begin{itemize}[noitemsep]
		\item The amount of the item available
		\item The type of item
		\item A description
		\item If the item possesses any antiquarian value
		\item A minimum price decided by the owner
		\item The date when brought in
		\item Name and address of the owner plus identification
		\item Planned auction date
		\item Distinguishing features
	\end{itemize}
	\item The system updates the item with the given information.
	\item The system returns to the user whether it was successful.
\end{enumerate}
\textsl{Extensions}
\begin{itemize}[noitemsep]
	\item If unsucessful, the system database remains unchanged; no item or other information was added or changed.
\end{itemize}
\textsl{System Sequence Diagram}
\begin{figure}[H]
	\centering
	\includegraphics[scale=1]{SD-bb-update.pdf}
	\caption*{Interactions displayed in a System Sequence Diagram in blackbox format}
\end{figure}
\subsection*{Grey Box Sequence Diagram - (A)}
\begin{figure}[H]
	\centering
	\includegraphics[scale=.9]{uml/SD-gb-update.pdf}
\end{figure}
\subsection*{White Box Sequence Diagram - (A)}
\begin{figure}[H]
	\centering
	\includegraphics[scale=.85]{uml/SD-wb-update.pdf}
\end{figure}

\subsection{Deleting - (C1)}
\subsubsection*{Analysis}
\begin{tabular}{@{}l l}
\textbf{Scope}:&The AuctionHouse\textsuperscript{TM} automated administration system\\
\textbf{Level}:&User goal\\
\textbf{Primary Actor}:&Auctioneer\\
\textbf{Stakeholders and Interests}:&\begin{tabular}[t]{@{}l}Auctioneer: Wants to remove items from the item list;
\\Owner: Does not want to display non-existent items to the viewers;
\\Viewer: Wants to see only existent items and (if applicable) their actual quantity.  \end{tabular}\\
\textbf{Preconditions}:&\begin{tabular}[t]{@{}l}Actor is identified and authenticated.\\Actor is authorized to access this use case.\end{tabular}\\
\textbf{Postconditions}:&\begin{tabular}[t]{@{}l}The item list has been updated and does not contain the removed item. \end{tabular}\\
\textbf{Special requirements}:&\begin{tabular}[t]{@{}l}The item should be identifiable (perhaps with an unique identification number).\end{tabular}\\
\textbf{Frequency of occurence}:&Very frequent after each auction (once per month), otherwise rarely to even none at all.\\
\end{tabular}\\\\
\textsl{Main Success Scenario}
\begin{enumerate}[noitemsep]
	\item The user enters the deletion mode for items.
	\item An empty \textit{selectedForDeletion} list is created.
	\item The system displays a filterable list of all items.
	\item The user filters the list using the following items characteristics:
	\begin{itemize}[noitemsep]
		\item The amount of the item available
		\item The type of item
		\item A description
		\item If the item possesses any antiquarian value
		\item A minimum price decided by the owner
		\item The date when brought in
		\item Name and address of the owner plus identification
		\item Planned auction date
		\item Distinguishing features
	\end{itemize}
	\item The user selects an item from the resultant filtered (shrinked) list.
	\item If the item is already in the \textit{selectedForDeletion} list remove it and jump to 9.
	\item The systems prompts the user whether the item is surely removed from the stockroom.
	\item If the user answers yes - add the item to the \textit{selectedForDeletion} list, otherwise do nothing.
	\item If the user wants to modify the \textit{selectedForDeletion} list, repeat steps 3-9.
	\item Remove all items in the \textit{selectedForDeletion} list from the full item list and display the number of overall deleted items.
\end{enumerate}

\textsl{System Sequence Diagram}\\\\
\begin{figure}[H]
	\centering
	\includegraphics[scale=1]{uml/SD-bb-delete.pdf}
	\caption*{Interactions displayed in a System Sequence Diagram defined by the MSS and its extensions in blackbox format.}
\end{figure}

\subsection{Create a Search Service Request - (A2)}
\subsubsection*{Analysis}
\begin{tabular}{@{}l l}
\textbf{Scope}:&The AuctionHouse\textsuperscript{TM} automated administration system\\
\textbf{Level}:&User goal\\
\textbf{Primary Actor}:&Owner\\
\textbf{Stakeholders and Interests}:&\begin{tabular}[t]{@{}l}Owner: wants to register the search request.\\Buyers: want their search request to be registered by the owner.\end{tabular}\\
\textbf{Preconditions}:&User is identified and authenticated.\\
\textbf{Postconditions}:&\begin{tabular}[t]{@{}l}The search request is registered in the system.\\The buyer is notified of confirmation or failure.\end{tabular}\\
\textbf{Special requirements}:&A yearly fee of €20 must be paid for the service to process the request. The system should check whether this is in order.\\
\textbf{Frequency of occurence}:& \begin{tabular}[t]{@{}l}TODO\end{tabular}
\end{tabular}\\\\
\textsl{Main Success Scenario}
\begin{enumerate}[noitemsep]
	\item The user starts the `file search request' transaction for a buyer with the system, having all parameters of the user to be added ready.
	\item The system confirms the buyer has paid the yearly fee
	\item The system confirms the interaction and asks the user to provide an item catalogue and a maximum price.
	\item The user provides these parameters.
	\item The system creates the search request with the item catalogue,the maximum price and the buyer interested and adds it to the list. The system also generates a Search ID for reference.
	\item The system informs the buyer that the request has been made.
	\item The system returns to the user with a confirmation message containing the Search ID, or a failure.
\end{enumerate}
\textsl{Extensions}
\begin{itemize}[noitemsep]
	\item If the yearly fee has not been paid, the system will return a failure and ends the transaction.
	\item When a failure is returned, the system state remains unchanged; no search request or other parameters was added.
	\item When not all required parameters were provided, the system will first ask the user to provide it until all parameters is provided.
\end{itemize}
\textsl{System Sequence Diagram}
\begin{figure}[H]
	\centering
	\includegraphics[scale=1]{SD-bb-createsearch.pdf}
	\caption*{Interactions displayed in a System Sequence Diagram defined by the MSS and its extensions in blackbox format}
\end{figure}


\section{Domain Model}
\begin{figure}[H]
	\centering
	\includegraphics[scale=.9]{uml/domainmodelUPD3.pdf}
\end{figure}
% arrows for domain models to copy: ► ◄ ▲ ▼

% Things to add:
% Search Request, Book (maybe multiple kinds and a catalogue, well see)
The domain model is a universal, graphical representation of the relevant concepts and their attributes. It shows how all concepts relate to eachother.\\
For example, you can see that items are `used' by a number of sources, such as buyers, sellers, the auctioneer and the purchasing agent. Items are also described by a catalogue: a general form of the item to distinguish the individual item from the `item' as it is described. These catalogues are showed in an item list, which is just the collection of item catalogues available for purchase. The item list can in turn be viewed by the public.\\
Resale items are instances of catalogue items. These resale items specify an item that is often bought for resale. For example, books, which are in turn an instance of a resale item, are often bought up by small second-hand booksellers who offer the books for sale for more interested customers. To avoid selling expesive books for low prices, their value is estimated, which determines if they are sold to bookstores or kept in the storeroom.\\
We also see that all staff is an instance of the staff class. Any staff is then managed by the owner of The AuctionHouse\textsuperscript{TM}. This does not only make sense for our customer, but also for a future implementation, even though that is not (yet) our concern.\\
We see that a couple of parties can register something to the system. One is the secretary can add potential buyers and sellers to the system, granting them more permissions than the general public. Another is the purchasing agent, who can, after evaluation, add items to the system, making them available for viewing by the public.\\
What remains is the ``Search System Request''. This is an entry for the search system, which is a service provided by the owner. The idea is that buyers can request to search for an item in The AuctionHouse\textsuperscript{TM}, or in a select group of other auctionhouses. The buyer can specify a maximum price he is willing to offer for the item. Whenever the owner find the item, he charges the buyer an amount depending on the price of the item. The service costs 20 euros a year.\\
The owner is the only one managing this service, so he is the only party who needs to interact with the requests.
\section{Class Diagram}
\section{References}
\subsection*{Concact with the customer}
\begin{enumerate}[1.]
	\item \label{conv1}\textsl{Us:}\\
	Dear John,\\

I noticed you forgot to inform us about the frequency of occurrence of the search request. It is vital for the development stage for us to know how often per month a search request is filed (on average).

Regards,\\
Your trusty development team (group 9)\\
\vspace{10pt}\\
\textsl{John:}\\
Dear people from group 9,\\

The frequency of Search request varies a lot. Sometimes we have 3 on one day, sometimes only one in a whole week. On average, I would say we have approx 12 per month.
\end{enumerate}
% \section{About Possible Implementation (optional)}

\end{document}