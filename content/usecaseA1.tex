\subsubsection*{Analysis}
\begin{tabular}{@{}l l}
\textbf{Scope}:&The AuctionHouse\textsuperscript{TM} automated administration system\\
\textbf{Level}:&User goal\\
\textbf{Primary Actor}:&Purchasing Agent\\
\textbf{Stakeholders and Interests}:&\begin{tabular}[t]{@{}l}Purchasing Agent: wants to enter parameters easily and quickly\\Seller: wants his goods to be visible to potential buyers\end{tabular}\\
\textbf{Preconditions}:&Purchasing agent is identified and authenticated.\\
\textbf{Postconditions}:&\begin{tabular}[t]{@{}l}The item has been added to the system, or a failure was returned.\\The product has been registered for an upcoming auction.\\The item is (or will become) visible for potential buyers\\and the otherwise authorized.\end{tabular}\\
\textbf{Special requirements}:&A list of categories and a list of future auction dates needs to available in the system\\
\textbf{Frequency of occurence}:&Depedent on the amount of sellers and auctions per month. Since there is 1 auction per month. If we expect about 30 sellers, then there are 30 occurences per month, so about 1 per day.
\end{tabular}\\\\
\textsl{Main Success Scenario}
\begin{enumerate}[noitemsep]
	\item The user starts the `add item' transaction with the system, having all parameters of the item to be added ready.
	\item The system provides the user with a list of categories to choose from, and a list of future planned aution dates.
	\item The user provides all necessary parameters and chooses a category for the item, and an auction date. The parameters to provide are:
	\begin{itemize}[noitemsep]
		\item The amount of the item available
		\item The type of item
		\item A description
		\item If the item possesses any antiquarian value
		\item A minimum price decided by the owner
		\item The date when brought in
		\item Name and address of the owner plus identification
		\item Planned auction date
		\item Distinguishing features
	\end{itemize}
	\item The system adds the item with all provided parameters to the list and generates an Item ID unique to the item.
	\item The system returns to the user with a confirmation message containing the generated item ID, or a failure.
\end{enumerate}
\textsl{Extensions}
\begin{itemize}[noitemsep]
	\item When a failure is returned, the system state remains unchanged; no item or other parameters was added.
	\item When not all required parameters were provided, the system will first ask the user to provide it until all parameters is provided.
\end{itemize}
\textsl{System Sequence Diagram}
\begin{figure}[H]
	\centering
	\includegraphics[scale=1]{SD-bb-create.pdf}
	\caption*{Interactions displayed in a System Sequence Diagram defined by the MSS and its extensions in blackbox format}
\end{figure}