A use case is a textual representation of what a certain interaction with the system might look like. A use case is described as an interaction between the actor or user and the system. First, description is provided in the form of a pointwise table. The points are
\begin{itemize}
	\item Scope: The system (can be digital or non-digital) to which this use case belongs.
	\item Level: Either ``user-goal'' or ``subfunction''. This indicates if this use case is a direct goal of the user, or if the use case describes a prerequisite for another use case. In this report, ``user-goal'' use cases will be complemented with a System Sequence diagram in black-box, gray-box and white-box format for completeness, whereas ``subfunction'' use cases will only contain the description and Main Success Scenario (since they belong to a ``user-goal'' use case, which already has the SDDs).
	\item Primary Actor: The main actor that interacts with the system, often the actor directly interfacing with the system.
	\item Stakeholders and Interests: Other parties that have interests in the results of this use case, and what those interests are. Actors and stakeholders do not neccesarily have to be human. A non-human stakeholder can be for example a part of the system that requires a certain condition to be invariant at all times in order to function as expected. In that case, that part of the system is a stakeholder and has interest in that invariant to be true at all times.
	\item Preconditions: Certain conditions that must hold before this use case can be executed.
	\item Postconditions: Conditions that will be true after this use case, often directly related to the stakeholders and their interests.
	\item Special Requirements: Non-functional requirements to the system (such as technical capabilities, or the availability of certain digital information).
	\item Frequency of Occurrence: How often it is expected this use case shall be executed.
\end{itemize}
With this as a backbone, the following use cases have been contructed.\\\\

\underline{Notice:} Some of the terms and structure are derived from the book by Larman, in conbination with knowledge gained during lectures.